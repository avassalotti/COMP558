\documentclass{article}
\usepackage{amsmath,amssymb,amsthm}
\usepackage{mathpazo}
\author{Alexandre Vassalotti \quad \'{E}ric Renaud-Houde}
\title{COMP 558: Final Project Report \\
  \large \textbf{Surface reconstruction using the level set method}
}
\date{27 April 2012}

\begin{document}
\maketitle
\section{Introduction}
Modelling surfaces from unorganized set of points, or point clouds, is a long
standing problem in the computer vision community. Indeed, problem is known to
be very challenging in three and higher dimensions. Furthermore, the problem
is ill-posed which means there is not a unique solution. When the point cloud
is dense enough and the topology of the surface is not complicated, a simple
solution could be to perform a triangulation of points. However, even in this
context ambiguities can arises and lead to non desirable surface
reconstructions.

A desirable reconstruction method should be able to deal with irregularities
caused by noise and non-uniformity of the data collected. A method should also
be able to deal with complex surface topologies as well. In addition, the
reconstructed surface should be representative of the point could data.



\section{Framework}

\paragraph{Discretization}

Given that the surface motion equation for $\phi$ is defined as a partial
differential equation, we can solve it using iterative schemes for PDEs. As most
differential equations which might appear fairly innocuous at first, solving
them numerically can be quite challenging.

Given an initial value $\phi_0$, our first intution might be to apply the
first-order Euler method as follows:
\begin{align}
  \frac{\partial \phi}{\partial t} + F \|\nabla \phi\| &= 0 \\
  \frac{\phi^{t+1} - \phi^{t}}{\Delta t} &=  -F \|\nabla \phi\| \\
  \phi^{t+1} &= \phi^{t} - \Delta t \; F \|\nabla \phi\| 
\end{align}

However, not only does the Euler method often suffers from stability problems,
it is used in the context of \textit{ordinary differential equations}, not PDEs.
In that case, a different numerical procedure is needed. 

\subparagraph{Upwind Scheme}
One approach defined for the initial value formulation by
Sethian\cite{sethian1999level} \cite{sethian1999advancing} is called the upwind
scheme.  Instead of using central differences for computing the partial
derivatives, it uses both a forward or a backward difference.


\bibliographystyle{amsplain}
\bibliography{vision}

\end{document}
